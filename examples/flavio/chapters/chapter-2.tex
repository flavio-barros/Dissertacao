% -*- coding: utf-8; -*-

\chapter{Provas Formais}

Este capítulo apresenta os principais conceitos relacionados às provas formais.

\section{Provas Formais \textit{vs} Provas Informais}

Segundo o dicionário \textit{Michaelis} da língua portuguesa, uma prova é "Aquilo que demonstra a veracidade de uma afirmação ou de um fato; confirmação, comprovação, evidência". Na matemática, as provas são utilizadas para certificar e comunicar o conhecimento entre especialistas. Provas matemáticas existem desde a Grécia antiga, no entanto, sua forma e organização mudaram bastante quando comparadas às provas matemáticas atuais.

O conceito de prova formal foi elaborado entre o final do século XIX e o início do século XX, o estilo de prova matemática que era utilizado antes desse período é chamado de prova informal. Uma \textit{prova informal} é expressa em linguagem natural e, possivelmente, contém símbolos e figuras \cite{1}. Esse tipo de prova tem o objetivo de convencer o leitor que a proposição matemática em questão é verdadeira ou falsa através da exposição de uma sequência de argumentos intuitivamente encadeados. Com o objetivo de favorecer a compreensão, algumas informações básicas e passos óbvios de raciocínio não são adicionados à prova, o que deixa algumas lacunas na argumentação.

Apesar de não ser matemático profissional, o filósofo alemão Immanuel Kant (1724-1804) foi o primeiro a perceber um problema no processo de construção das provas informais. Sem aprofundamentos nos aspectos filosóficos, o Problema de Kant pode ser enunciado da seguinte maneira: faz-se necessário que os conceitos envolvidos em uma prova estejam ligados, sem que existam lacunas na argumentação (Jody Azzouni). Para Kant, as provas matemáticas deveriam ficar restrita aos conceitos analíticos.