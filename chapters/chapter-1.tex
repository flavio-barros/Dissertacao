% -*- coding: utf-8; -*-

\chapter{Introdução}

Provas matemáticas existem desde a Grécia antiga e têm o objetivo de corroborar a veracidade de uma afirmação, além de transmitir conhecimento entre especialistas através do tempo. As provas matemáticas utilizadas até o final do século XIX, conhecidas como provas informais, são geralmente construídas em linguagem natural e não possuem uma forma precisa, sendo estruturadas de acordo com a vontade do autor. Apesar de serem bastante utilizadas, as provas informais têm um importante problema: inexiste um método efetivo de verificação de erros. Entre o final do século XIX e o início do século XX, alguns matemáticos e lógicos abordaram diferentes maneiras para fundamentar a matemática através de um aparato teórico que permitisse uma checagem de erros. Entre as várias contribuições para a lógica matemática, surgiu o ramo da \textit{teoria da prova} juntamente com o conceito de prova formal.

Ao contrário das provas informais, as provas formais são estruturadas através de rigorosas regras de construção. Uma prova formal é a derivação de uma sentença (conclusão) a partir de outras sentenças (axiomas e hipóteses), tal derivação é uma sequência de sentenças, onde cada elemento da sequência é uma hipótese ou axioma, ou é resultado da aplicação de uma regra de inferência de um sistema dedutivo. Cada sentença de uma prova formal, também chamada de fórmula, é um elemento de uma linguagem formal, linguagens da Lógica Proposicional e da Lógica de Primeira Ordem são exemplos de linguagens formais.

Além da utilização puramente teórica na lógica matemática, provas formais podem ser utilizadas para diversas finalidades em diferentes contextos. No desenvolvimento de \textit{softwares}, provas podem ser necessárias para validar o funcionamento de porções de código. Na fabricação de \textit{hardwares}, um projeto de circuito pode ser validado através da prova de uma fórmula que o descreve. Em alguns casos, tais provas podem ser grandes, possuindo tamanhos exponenciais em relação ao tamanho da conclusão, o que aumenta significativamente a complexidade da construção e demanda uma automação.

A Prova Automática de Teoremas (\textit{Automated Theorem Proving} - ATP) é a área da Ciência da Computação que utiliza programas de computador para a geração automatizada ou semiautomatizada de provas. No entanto, um provador de teoremas ainda pode gerar provas muito grandes. O tamanho de uma prova pode prejudicar sua utilização prática, visto que a extração de alguma informação útil ao contexto pode se tornar inviável, além de que manipular grandes volumes de dados pode acarretar problemas de implementação para o provador. Esses problemas reforçam a importância do esforço para compressão das provas geradas e/ou na geração de provas já comprimidas.

Além de possíveis problemas de implementação nos provadores, o tamanho das provas possui algumas importantes implicações teóricas na área da complexidade computacional. Statman mostrou que o problema de determinar se uma fórmula é uma tautologia ($TAUT$) da Lógica Proposicional Intuicionista e do fragmento puramente implicacional da Lógica Minimal (M$\supset$) é PSPACE-Completo \cite{STATMAN197967}. M$\supset$ é capaz de simular a Lógica Proposicional Intuicionista através de uma tradução polinomial. Qualquer lógica proposicional com um sistema de dedução natural que satisfaça o princípio da subfórmula também possui seu respectivo $TAUT$ em $PSPACE$ \cite{Haeusler2014}. Um sistema de dedução natural que satisfaz o princípio da subfórmula é capaz de gerar provas onde cada ocorrência de fórmula é uma subfórmula da conclusão ou é uma subfórmula de alguma hipótese. Saber se $TAUT$ da M$\supset$ possui um certificado polinomial para qualquer tautologia, i.e, saber se qualquer tautologia possui uma prova de tamanho polinomialmente limitado em relação ao tamanho da conclusão, está relacionado com saber se $NP = PSPACE$.

Provas em dedução natural podem ser representadas em diferentes formatos. No estilo de Gentzen-Prawitz, as provas possuem o formato de uma árvore, onde as fórmulas são representadas pelos nós, e as regras e os rótulos de descarte são representados pelas arestas. No estilo de Ja{\'s}kowski-Fitch, as provas são sequências de passos numerados seguidos pela identificação da regra e sua referida justificativa. O tamanho de uma prova pode ser aferido a partir de diferentes pontos de vista. A quantidade de nós (Gentzen-Prawitz), a quantidade de linhas (Ja{\'s}kowski-Fitch) e até a quantidade de símbolos podem ser utilizados para mensurar o tamanho de uma prova. Para a complexidade computacional, o tamanho de uma prova é a quantidade de símbolos de sua representação.

É bem conhecido que provas podem ser muito grandes. Na M$\supset$, algumas fórmulas possuem provas em dedução natural com tamanhos com limite inferior exponencial \cite{haeusler2015many}. A redução do tamanho de provas pode ser realizada, principalmente, através de duas abordagens: gerar provas já comprimidas através de um cálculo de dedução natural que seja capaz de gerar provas menores que a dedução natural usual; comprimir provas em dedução natural que já tenham sido geradas.

Em \cite{NDcPaleo, paleo2015implementation}, Paleo propõe um cálculo de dedução natural para a M$\supset$ utilizando \textit{deep inference}, que permite aplicar as regras de inferência diretamente nas subfórmulas. Para algumas fórmulas, esse cálculo pode gerar provas quadraticamente menores que a dedução natural usual.

Em \cite{GordeevH16}, Gordeev e Haeusler propõem o método de Compressão Horizontal que permite reduzir o tamanho das provas através da fusão de nós com fórmulas idênticas que estejam no mesmo nível na árvore de derivação (estilo Gentzen-Prawitz). Essa técnica de compressão é utilizada como uma ferramenta para a prova da conjectura $NP = PSPACE$. O objetivo da compressão é que a prova compactada possua tamanho polinomialmente limitado em relação ao tamanho da conclusão.

A Compressão Horizontal e outras técnicas de compressão de provas para outros sistemas dedutivos, \cite{vyskovcil2010automated, amjad2008data, boudou2014skeptik}, utilizam a redundância de dados nas representações para obter a redução do espaço necessário para representar as provas. Essa característica também é a base para inúmeras técnicas tradicionais de compressão de dados, e.g., codificação de Huffman.

Esta dissertação de mestrado se propõe a realizar um estudo comparativo empírico entre técnicas de compressão de provas na M$\supset$ em dedução natural. A revisão da literatura identificou apenas duas técnicas de compressão com essa característica (dedução natural da M$\supset$). A primeira é a Dedução Natural Contextual (DNc) proposta em 2013 \cite{NDcPaleo} com experimentos reportados em 2015 \cite{paleo2015implementation}, que demonstraram que a técnica é capaz de gerar provas menores que a dedução natural usual em alguns casos. A segunda é a compressão horizontal de provas proposta em 2016 \cite{GordeevH16} para comprimir provas na M$\supset$ de tautologias arbitrárias com garantia que a compactação gera provas de tamanho polinomialmente limitado em relação ao tamanho da conclusão.

Apresentamos a primeira implementação da Compressão Horizontal juntamente com os resultados da aplicação das técnicas de compressão de provas e dados para um conjunto de provas na M$\supset$.

No Capítulo \ref{cap:prov_form}, introduzimos os principais conceitos relativos a provas formais utilizados no restante do trabalho. Iniciamos com um pequeno resumo histórico sobre o surgimento das provas formais. Apresentamos o fragmento puramente implicacional da lógica minimal, detalhando sua linguagem, sistema de dedução natural e semântica. Mostramos como uma prova em dedução natural de uma mesma fórmula pode possuir diferentes representações

No Capítulo \ref{cap:comp_prov_dado}, apresentamos em detalhes a Compressão Horizontal e a a codificação de Huffman. Exemplificamos como a Compressão Horizontal atua na compressão das provas no Capítulo \ref{cap:aplicacao_tec} e detalhamos passo a passo a compressão de uma prova. No Capítulo \ref{cap:impl_exp}, apresentamos o compressor de provas \textit{compressing}, que implementa a Compressão Horizontal. Detalhamos os principais aspectos do projeto e justificamos algumas decisões de projeto na implementação do compressor.

O Capítulo \ref{cap:experimentos} apresenta os resultados dos experimentos dos algoritmos da Compressão Horizontal e codificação de Huffman, detalhando a definição do conjunto de provas utilizadas e os \textit{softwares} utilizados direta ou indiretamente na experimentação. Concluímos o trabalho no Capítulo \ref{cap:conc_trab}, no qual avaliamos os objetivos inicialmente estabelecidos e os resultados obtidos, destacamos as contribuições do trabalho e exploramos possibilidades de trabalhos futuros.